\PassOptionsToPackage{unicode=true}{hyperref} % options for packages loaded elsewhere
\PassOptionsToPackage{hyphens}{url}
%
\documentclass[]{article}
\usepackage{lmodern}
\usepackage{amssymb,amsmath}
\usepackage{ifxetex,ifluatex}
\usepackage{fixltx2e} % provides \textsubscript
\ifnum 0\ifxetex 1\fi\ifluatex 1\fi=0 % if pdftex
  \usepackage[T1]{fontenc}
  \usepackage[utf8]{inputenc}
  \usepackage{textcomp} % provides euro and other symbols
\else % if luatex or xelatex
  \usepackage{unicode-math}
  \defaultfontfeatures{Ligatures=TeX,Scale=MatchLowercase}
\fi
% use upquote if available, for straight quotes in verbatim environments
\IfFileExists{upquote.sty}{\usepackage{upquote}}{}
% use microtype if available
\IfFileExists{microtype.sty}{%
\usepackage[]{microtype}
\UseMicrotypeSet[protrusion]{basicmath} % disable protrusion for tt fonts
}{}
\IfFileExists{parskip.sty}{%
\usepackage{parskip}
}{% else
\setlength{\parindent}{0pt}
\setlength{\parskip}{6pt plus 2pt minus 1pt}
}
\usepackage{hyperref}
\hypersetup{
            pdftitle={HW\_3/9},
            pdfauthor={Larin Anton},
            pdfborder={0 0 0},
            breaklinks=true}
\urlstyle{same}  % don't use monospace font for urls
\usepackage[margin=1in]{geometry}
\usepackage{color}
\usepackage{fancyvrb}
\newcommand{\VerbBar}{|}
\newcommand{\VERB}{\Verb[commandchars=\\\{\}]}
\DefineVerbatimEnvironment{Highlighting}{Verbatim}{commandchars=\\\{\}}
% Add ',fontsize=\small' for more characters per line
\usepackage{framed}
\definecolor{shadecolor}{RGB}{248,248,248}
\newenvironment{Shaded}{\begin{snugshade}}{\end{snugshade}}
\newcommand{\AlertTok}[1]{\textcolor[rgb]{0.94,0.16,0.16}{#1}}
\newcommand{\AnnotationTok}[1]{\textcolor[rgb]{0.56,0.35,0.01}{\textbf{\textit{#1}}}}
\newcommand{\AttributeTok}[1]{\textcolor[rgb]{0.77,0.63,0.00}{#1}}
\newcommand{\BaseNTok}[1]{\textcolor[rgb]{0.00,0.00,0.81}{#1}}
\newcommand{\BuiltInTok}[1]{#1}
\newcommand{\CharTok}[1]{\textcolor[rgb]{0.31,0.60,0.02}{#1}}
\newcommand{\CommentTok}[1]{\textcolor[rgb]{0.56,0.35,0.01}{\textit{#1}}}
\newcommand{\CommentVarTok}[1]{\textcolor[rgb]{0.56,0.35,0.01}{\textbf{\textit{#1}}}}
\newcommand{\ConstantTok}[1]{\textcolor[rgb]{0.00,0.00,0.00}{#1}}
\newcommand{\ControlFlowTok}[1]{\textcolor[rgb]{0.13,0.29,0.53}{\textbf{#1}}}
\newcommand{\DataTypeTok}[1]{\textcolor[rgb]{0.13,0.29,0.53}{#1}}
\newcommand{\DecValTok}[1]{\textcolor[rgb]{0.00,0.00,0.81}{#1}}
\newcommand{\DocumentationTok}[1]{\textcolor[rgb]{0.56,0.35,0.01}{\textbf{\textit{#1}}}}
\newcommand{\ErrorTok}[1]{\textcolor[rgb]{0.64,0.00,0.00}{\textbf{#1}}}
\newcommand{\ExtensionTok}[1]{#1}
\newcommand{\FloatTok}[1]{\textcolor[rgb]{0.00,0.00,0.81}{#1}}
\newcommand{\FunctionTok}[1]{\textcolor[rgb]{0.00,0.00,0.00}{#1}}
\newcommand{\ImportTok}[1]{#1}
\newcommand{\InformationTok}[1]{\textcolor[rgb]{0.56,0.35,0.01}{\textbf{\textit{#1}}}}
\newcommand{\KeywordTok}[1]{\textcolor[rgb]{0.13,0.29,0.53}{\textbf{#1}}}
\newcommand{\NormalTok}[1]{#1}
\newcommand{\OperatorTok}[1]{\textcolor[rgb]{0.81,0.36,0.00}{\textbf{#1}}}
\newcommand{\OtherTok}[1]{\textcolor[rgb]{0.56,0.35,0.01}{#1}}
\newcommand{\PreprocessorTok}[1]{\textcolor[rgb]{0.56,0.35,0.01}{\textit{#1}}}
\newcommand{\RegionMarkerTok}[1]{#1}
\newcommand{\SpecialCharTok}[1]{\textcolor[rgb]{0.00,0.00,0.00}{#1}}
\newcommand{\SpecialStringTok}[1]{\textcolor[rgb]{0.31,0.60,0.02}{#1}}
\newcommand{\StringTok}[1]{\textcolor[rgb]{0.31,0.60,0.02}{#1}}
\newcommand{\VariableTok}[1]{\textcolor[rgb]{0.00,0.00,0.00}{#1}}
\newcommand{\VerbatimStringTok}[1]{\textcolor[rgb]{0.31,0.60,0.02}{#1}}
\newcommand{\WarningTok}[1]{\textcolor[rgb]{0.56,0.35,0.01}{\textbf{\textit{#1}}}}
\usepackage{graphicx,grffile}
\makeatletter
\def\maxwidth{\ifdim\Gin@nat@width>\linewidth\linewidth\else\Gin@nat@width\fi}
\def\maxheight{\ifdim\Gin@nat@height>\textheight\textheight\else\Gin@nat@height\fi}
\makeatother
% Scale images if necessary, so that they will not overflow the page
% margins by default, and it is still possible to overwrite the defaults
% using explicit options in \includegraphics[width, height, ...]{}
\setkeys{Gin}{width=\maxwidth,height=\maxheight,keepaspectratio}
\setlength{\emergencystretch}{3em}  % prevent overfull lines
\providecommand{\tightlist}{%
  \setlength{\itemsep}{0pt}\setlength{\parskip}{0pt}}
\setcounter{secnumdepth}{0}
% Redefines (sub)paragraphs to behave more like sections
\ifx\paragraph\undefined\else
\let\oldparagraph\paragraph
\renewcommand{\paragraph}[1]{\oldparagraph{#1}\mbox{}}
\fi
\ifx\subparagraph\undefined\else
\let\oldsubparagraph\subparagraph
\renewcommand{\subparagraph}[1]{\oldsubparagraph{#1}\mbox{}}
\fi

% set default figure placement to htbp
\makeatletter
\def\fps@figure{htbp}
\makeatother


\title{HW\_3/9}
\author{Larin Anton}
\date{12/18/2020}

\begin{document}
\maketitle

\hypertarget{ux438ux434ux437-3-ux441ux442ux430ux442ux430ux43d}{%
\subsection{ИДЗ 3
Статан}\label{ux438ux434ux437-3-ux441ux442ux430ux442ux430ux43d}}

\hypertarget{ux438ux434ux437-9-ux43cux430ux442ux43fux430ux43aux435ux442ux44b}{%
\subsection{ИДЗ 9
Матпакеты}\label{ux438ux434ux437-9-ux43cux430ux442ux43fux430ux43aux435ux442ux44b}}

\#Ход работы \#\#Графические представление

\begin{Shaded}
\begin{Highlighting}[]
\KeywordTok{plot}\NormalTok{(x,y,}\DataTypeTok{main=}\StringTok{"Result"}\NormalTok{)}
\end{Highlighting}
\end{Shaded}

\includegraphics{Rmd_knit_files/figure-latex/unnamed-chunk-1-1.pdf}
\(\lim_{x \to 0}\frac{\sin(x)}{x} = 1\)

Линейная регрессионная модель: \(y = \beta_0 + \beta_1*x\)

\hypertarget{section}{%
\subsection{1}\label{section}}

МНК оценка параметров сдвига \(\beta_0\) и масштаба \(\beta_1\)

\begin{Shaded}
\begin{Highlighting}[]
\NormalTok{n<-}\KeywordTok{length}\NormalTok{(y)}
\NormalTok{x0<-}\KeywordTok{array}\NormalTok{(}\DecValTok{1}\NormalTok{,}\DataTypeTok{dim=}\NormalTok{n)}
\NormalTok{X<-}\KeywordTok{t}\NormalTok{(}\KeywordTok{matrix}\NormalTok{(}\KeywordTok{c}\NormalTok{(x0,x),}\DataTypeTok{nrow=}\NormalTok{n,}\DataTypeTok{ncol=}\DecValTok{2}\NormalTok{))}
\NormalTok{Y<-}\KeywordTok{as.matrix}\NormalTok{(y)}
\NormalTok{S<-X}\OperatorTok\KeywordTok{t}\NormalTok{(X)}
\NormalTok{S1<-}\KeywordTok{solve}\NormalTok{(S)}
\NormalTok{bhat<-S1}\OperatorTok\NormalTok{X}\OperatorTok\NormalTok{Y}
\end{Highlighting}
\end{Shaded}

\hypertarget{ux440ux435ux437ux443ux43bux44cux442ux430ux442}{%
\paragraph{Результат:}\label{ux440ux435ux437ux443ux43bux44cux442ux430ux442}}

\(\beta_0 = 5.5889333\)\\
\(\beta_1 = 0.5546667\)

Нарисуем полученную регрессионную зависимость

\includegraphics{Rmd_knit_files/figure-latex/unnamed-chunk-4-1.pdf}

\hypertarget{section-1}{%
\subsection{2}\label{section-1}}

Построение несмещенной оценки дисперсии

\begin{Shaded}
\begin{Highlighting}[]
\NormalTok{  res<-Y}\OperatorTok{-}\KeywordTok{t}\NormalTok{(X)}\OperatorTok\KeywordTok{as.matrix}\NormalTok{(bhat)}
\NormalTok{SS<-}\KeywordTok{sum}\NormalTok{(res}\OperatorTok{^}\DecValTok{2}\NormalTok{)}
\NormalTok{s2<-SS}\OperatorTok{/}\NormalTok{(n}\DecValTok{-2}\NormalTok{)}
\end{Highlighting}
\end{Shaded}

Результат: \(4.0704077\)

Построение гистограммы с шагом h = 1.4 на базе ошибок

\includegraphics{Rmd_knit_files/figure-latex/unnamed-chunk-6-1.pdf}

Проверка гипотезы нормальности ошибок на уровне \(\alpha\) по \(\chi^2\)
\(H_0: Y-X^T \beta\sim(0,\sigma^2)\)

\begin{Shaded}
\begin{Highlighting}[]
\NormalTok{hh<-}\KeywordTok{hist}\NormalTok{(res,}\DataTypeTok{breaks=}\NormalTok{brk,}\DataTypeTok{plot=}\OtherTok{FALSE}\NormalTok{)}
\NormalTok{nu<-hh}\OperatorTok{$}\NormalTok{counts}
\NormalTok{breaks =}\StringTok{ }\NormalTok{hh}\OperatorTok{$}\NormalTok{breaks;}
\NormalTok{r =}\StringTok{ }\KeywordTok{length}\NormalTok{(breaks) }\OperatorTok{-}\StringTok{ }\DecValTok{1}
\NormalTok{l.b<-}\KeywordTok{length}\NormalTok{(brk)}
\NormalTok{csq0<-}\ControlFlowTok{function}\NormalTok{(s)\{}
  \ControlFlowTok{if}\NormalTok{ (s}\OperatorTok{>}\DecValTok{0}\NormalTok{)\{}
\NormalTok{    p<-}\KeywordTok{pnorm}\NormalTok{(brk[}\DecValTok{2}\OperatorTok{:}\NormalTok{l.b],}\DecValTok{0}\NormalTok{,s)}\OperatorTok{-}\KeywordTok{pnorm}\NormalTok{(brk[}\DecValTok{1}\OperatorTok{:}\NormalTok{(l.b}\DecValTok{-1}\NormalTok{)],}\DecValTok{0}\NormalTok{,s)}
    \KeywordTok{return}\NormalTok{(}\KeywordTok{sum}\NormalTok{((nu}\OperatorTok{-}\NormalTok{n}\OperatorTok{*}\NormalTok{p)}\OperatorTok{^}\DecValTok{2}\OperatorTok{/}\NormalTok{n}\OperatorTok{/}\NormalTok{p))}
\NormalTok{  \} }\ControlFlowTok{else}\NormalTok{ \{}
    \KeywordTok{return}\NormalTok{(}\OtherTok{Inf}\NormalTok{)}
\NormalTok{  \}}
\NormalTok{\}}
\NormalTok{csq.s<-}\KeywordTok{nlm}\NormalTok{(csq0,}\DataTypeTok{p=}\KeywordTok{sqrt}\NormalTok{(s2))}\OperatorTok{$}\NormalTok{minimum}
\NormalTok{pv<-}\KeywordTok{pchisq}\NormalTok{(csq.s,r }\OperatorTok{-}\StringTok{ }\DecValTok{2}\NormalTok{,}\DataTypeTok{lower.tail=}\OtherTok{FALSE}\NormalTok{)}
\end{Highlighting}
\end{Shaded}

Результат: FALSE

Оценим расстояние оценки до класса нормальных распределений оп
Колмогорову

\begin{Shaded}
\begin{Highlighting}[]
\NormalTok{kolm.stat<-}\ControlFlowTok{function}\NormalTok{(s)\{}
\NormalTok{  sres<-}\KeywordTok{sort}\NormalTok{(res)}
\NormalTok{  fdistr<-}\KeywordTok{pnorm}\NormalTok{(sres,}\DecValTok{0}\NormalTok{,s)}
  \KeywordTok{max}\NormalTok{(}\KeywordTok{abs}\NormalTok{(}\KeywordTok{c}\NormalTok{(}\DecValTok{0}\OperatorTok{:}\NormalTok{(n}\DecValTok{-1}\NormalTok{))}\OperatorTok{/}\NormalTok{n}\OperatorTok{-}\NormalTok{fdistr),}\KeywordTok{abs}\NormalTok{(}\KeywordTok{c}\NormalTok{(}\DecValTok{1}\OperatorTok{:}\NormalTok{n)}\OperatorTok{/}\NormalTok{n}\OperatorTok{-}\NormalTok{fdistr))}
\NormalTok{\}}
\NormalTok{ks.dist<-}\KeywordTok{nlm}\NormalTok{(kolm.stat,}\DataTypeTok{p=}\KeywordTok{sqrt}\NormalTok{(s2))}\OperatorTok{$}\NormalTok{minimum}
\end{Highlighting}
\end{Shaded}

Полученое расстояние: 0.0708533

\includegraphics{Rmd_knit_files/figure-latex/unnamed-chunk-9-1.pdf}

\hypertarget{section-2}{%
\subsection{3}\label{section-2}}

Построим ДИ для параметров с уровнем доверия \(1-\alpha (0.8)\)

\begin{Shaded}
\begin{Highlighting}[]
\NormalTok{C<-}\KeywordTok{diag}\NormalTok{(}\KeywordTok{c}\NormalTok{(}\DecValTok{1}\NormalTok{,}\DecValTok{1}\NormalTok{))}
\NormalTok{ph<-bhat}
\NormalTok{V<-}\KeywordTok{diag}\NormalTok{(S1)}
\NormalTok{xa<-}\KeywordTok{qt}\NormalTok{(}\DecValTok{1}\OperatorTok{-}\NormalTok{al}\OperatorTok{/}\DecValTok{2}\NormalTok{,n}\DecValTok{-2}\NormalTok{)}
\NormalTok{s1<-}\KeywordTok{sqrt}\NormalTok{(s2)}
\NormalTok{d<-xa}\OperatorTok{*}\NormalTok{s1}\OperatorTok{*}\KeywordTok{sqrt}\NormalTok{(V)}
\NormalTok{CI<-}\KeywordTok{data.frame}\NormalTok{(}\DataTypeTok{lw=}\NormalTok{ph}\OperatorTok{-}\NormalTok{d,}\DataTypeTok{up=}\NormalTok{ph}\OperatorTok{+}\NormalTok{d)}
\end{Highlighting}
\end{Shaded}

Для \(\beta_0: [4.7975981,6.3802685]\)\\
Для \(\beta_1: [0.1177249,0.9916084]\)

Доверительный эллипс можно вычислить как

\(A\alpha=\{x,y:((xy)^T-\beta)^T*(XX^T)^{-1}*((xy)^T-\beta)\le qs^2x\alpha\}\)\\
Где:

\(\beta=\)

\begin{verbatim}
##           [,1]
## [1,] 5.5889333
## [2,] 0.5546667
\end{verbatim}

\((XX^T)=\)

\begin{verbatim}
##             [,1]        [,2]
## [1,]  0.09111111 -0.04444444
## [2,] -0.04444444  0.02777778
\end{verbatim}

\hypertarget{section-3}{%
\subsection{4}\label{section-3}}

Гипотеза независимости Y от X: \(H0: \beta1=0\)\\
Критерий:\\
\(\Phi(x)=\begin{cases} 0, \alpha<PV \\ 1,\alpha>PV \end{cases}\)

Найдем статистику F-критерия и P-значение:

\begin{Shaded}
\begin{Highlighting}[]
\NormalTok{FST<-bhat[}\DecValTok{2}\NormalTok{]}\OperatorTok{^}\DecValTok{2}\OperatorTok{/}\NormalTok{V[}\DecValTok{2}\NormalTok{]}\OperatorTok{/}\NormalTok{s2}
\NormalTok{pv.f<-}\KeywordTok{pf}\NormalTok{(FST,}\DecValTok{1}\NormalTok{,n}\DecValTok{-2}\NormalTok{,}\DataTypeTok{lower.tail=}\OtherTok{FALSE}\NormalTok{)}
\end{Highlighting}
\end{Shaded}

Получим 2.7210012 и 0.1055659\\
Результат: FALSE

\hypertarget{section-4}{%
\subsection{5}\label{section-4}}

Добавим в модель член с \(X^2\):\\
\(Y=\beta_1+\beta_2x+\beta_3x^2+\varepsilon\)

Найдем МНК оценки

\begin{Shaded}
\begin{Highlighting}[]
\NormalTok{x0 <-}\StringTok{ }\KeywordTok{array}\NormalTok{(}\DecValTok{1}\NormalTok{, }\DataTypeTok{dim=}\NormalTok{n)}
\NormalTok{X <-}\StringTok{ }\KeywordTok{t}\NormalTok{(}\KeywordTok{matrix}\NormalTok{(}\KeywordTok{c}\NormalTok{(x0, x, x}\OperatorTok{^}\DecValTok{2}\NormalTok{), }\DataTypeTok{nrow=}\NormalTok{n, }\DataTypeTok{ncol=}\DecValTok{3}\NormalTok{))}
\NormalTok{Y <-}\StringTok{ }\KeywordTok{as.matrix}\NormalTok{(y)}
\NormalTok{S <-}\StringTok{ }\NormalTok{X}\OperatorTok\KeywordTok{t}\NormalTok{(X)}
\NormalTok{S1 <-}\StringTok{ }\KeywordTok{solve}\NormalTok{(S)}
\NormalTok{bht <-}\StringTok{ }\NormalTok{S1}\OperatorTok\NormalTok{X}\OperatorTok\NormalTok{Y}
\end{Highlighting}
\end{Shaded}

\(\beta1=4.7528712\)\\
\(\beta2=1.8515702\)\\
\(\beta3=-0.3777389\)

Нарисуем полученную регрессионную зависимость

\includegraphics{Rmd_knit_files/figure-latex/unnamed-chunk-15-1.pdf}

\hypertarget{section-5}{%
\subsection{6}\label{section-5}}

Несмещенная оценка дисперсии

\begin{Shaded}
\begin{Highlighting}[]
\NormalTok{res<-Y}\OperatorTok{-}\KeywordTok{t}\NormalTok{(X)}\OperatorTok\KeywordTok{as.matrix}\NormalTok{(bht)}
\NormalTok{SS<-}\KeywordTok{sum}\NormalTok{(res}\OperatorTok{^}\DecValTok{2}\NormalTok{)}
\NormalTok{s2<-SS}\OperatorTok{/}\NormalTok{(n}\DecValTok{-2}\NormalTok{)}
\end{Highlighting}
\end{Shaded}

Результат: 3.9820608

Гистограмма на базе ошибок

\begin{Shaded}
\begin{Highlighting}[]
\NormalTok{brk<-}\KeywordTok{seq}\NormalTok{(}\KeywordTok{min}\NormalTok{(res), }\KeywordTok{max}\NormalTok{(res) }\OperatorTok{+}\StringTok{ }\NormalTok{h, }\DataTypeTok{by=}\NormalTok{h)}
\KeywordTok{hist}\NormalTok{(res,}\DataTypeTok{breaks=}\NormalTok{brk)}
\end{Highlighting}
\end{Shaded}

\includegraphics{Rmd_knit_files/figure-latex/unnamed-chunk-17-1.pdf}

Проверка гипотезы нормальности ошибок на уровне \(\alpha\) по \(\chi^2\)

\begin{Shaded}
\begin{Highlighting}[]
\NormalTok{l.b<-}\KeywordTok{length}\NormalTok{(brk)}
\NormalTok{brk[}\DecValTok{1}\NormalTok{]<-}\StringTok{ }\OperatorTok{-}\OtherTok{Inf}
\NormalTok{brk[l.b]<-}\OtherTok{Inf}
\CommentTok{#r = length(breaks) - 1}
\NormalTok{hh<-}\KeywordTok{hist}\NormalTok{(res,}\DataTypeTok{breaks=}\NormalTok{brk,}\DataTypeTok{plot=}\OtherTok{FALSE}\NormalTok{)}
\NormalTok{nu<-hh}\OperatorTok{$}\NormalTok{counts}
\NormalTok{breaks =}\StringTok{ }\NormalTok{hh}\OperatorTok{$}\NormalTok{breaks;}
\NormalTok{r =}\StringTok{ }\KeywordTok{length}\NormalTok{(breaks) }\OperatorTok{-}\StringTok{ }\DecValTok{1}

\NormalTok{l.b<-}\KeywordTok{length}\NormalTok{(brk)}
\NormalTok{csq0<-}\ControlFlowTok{function}\NormalTok{(s)\{}
  \ControlFlowTok{if}\NormalTok{ (s}\OperatorTok{>}\DecValTok{0}\NormalTok{)\{}
\NormalTok{    p<-}\KeywordTok{pnorm}\NormalTok{(brk[}\DecValTok{2}\OperatorTok{:}\NormalTok{l.b],}\DecValTok{0}\NormalTok{,s)}\OperatorTok{-}\KeywordTok{pnorm}\NormalTok{(brk[}\DecValTok{1}\OperatorTok{:}\NormalTok{(l.b}\DecValTok{-1}\NormalTok{)],}\DecValTok{0}\NormalTok{,s)}
    \KeywordTok{return}\NormalTok{(}\KeywordTok{sum}\NormalTok{((nu}\OperatorTok{-}\NormalTok{n}\OperatorTok{*}\NormalTok{p)}\OperatorTok{^}\DecValTok{2}\OperatorTok{/}\NormalTok{n}\OperatorTok{/}\NormalTok{p))}
\NormalTok{  \} }\ControlFlowTok{else}\NormalTok{ \{}
    \KeywordTok{return}\NormalTok{(}\OtherTok{Inf}\NormalTok{)}
\NormalTok{  \}}
\NormalTok{\}}
\NormalTok{csq.s<-}\KeywordTok{nlm}\NormalTok{(csq0,}\DataTypeTok{p=}\KeywordTok{sqrt}\NormalTok{(s2))}\OperatorTok{$}\NormalTok{minimum}
\end{Highlighting}
\end{Shaded}

\begin{verbatim}
## Warning in nlm(csq0, p = sqrt(s2)): NA/Inf replaced by maximum positive value
\end{verbatim}

\begin{Shaded}
\begin{Highlighting}[]
\NormalTok{pv<-}\KeywordTok{pchisq}\NormalTok{(csq.s,r}\DecValTok{-3}\NormalTok{,}\DataTypeTok{lower.tail=}\OtherTok{FALSE}\NormalTok{)}
\end{Highlighting}
\end{Shaded}

Результат: FALSE

Оценка расстояния до нормального рапределения по Колмагорову

\begin{Shaded}
\begin{Highlighting}[]
\NormalTok{kolm.stat<-}\ControlFlowTok{function}\NormalTok{(s)\{}
\NormalTok{  sres<-}\KeywordTok{sort}\NormalTok{(res)}
\NormalTok{  fdistr<-}\KeywordTok{pnorm}\NormalTok{(sres,}\DecValTok{0}\NormalTok{,s)}
  \KeywordTok{max}\NormalTok{(}\KeywordTok{abs}\NormalTok{(}\KeywordTok{c}\NormalTok{(}\DecValTok{0}\OperatorTok{:}\NormalTok{(n}\DecValTok{-1}\NormalTok{))}\OperatorTok{/}\NormalTok{n}\OperatorTok{-}\NormalTok{fdistr),}\KeywordTok{abs}\NormalTok{(}\KeywordTok{c}\NormalTok{(}\DecValTok{1}\OperatorTok{:}\NormalTok{n)}\OperatorTok{/}\NormalTok{n}\OperatorTok{-}\NormalTok{fdistr))}
\NormalTok{\}}
\NormalTok{ks.dist<-}\KeywordTok{nlm}\NormalTok{(kolm.stat,}\DataTypeTok{p=}\KeywordTok{sqrt}\NormalTok{(s2))}\OperatorTok{$}\NormalTok{minimum}


\NormalTok{x2<-}\KeywordTok{c}\NormalTok{(}\DecValTok{0}\OperatorTok{:}\DecValTok{1000}\NormalTok{)}\OperatorTok{*}\NormalTok{(}\KeywordTok{max}\NormalTok{(res)}\OperatorTok{-}\KeywordTok{min}\NormalTok{(res))}\OperatorTok{/}\DecValTok{1000}\OperatorTok{+}\KeywordTok{min}\NormalTok{(res)}
\NormalTok{y2<-}\KeywordTok{pnorm}\NormalTok{(x2,}\DecValTok{0}\NormalTok{,}\KeywordTok{nlm}\NormalTok{(kolm.stat,}\DataTypeTok{p=}\KeywordTok{sqrt}\NormalTok{(s2))}\OperatorTok{$}\NormalTok{estimate)}
\end{Highlighting}
\end{Shaded}

\includegraphics{Rmd_knit_files/figure-latex/unnamed-chunk-20-1.pdf}

Полученое расстояние: 0.0677388

\hypertarget{section-6}{%
\subsection{7}\label{section-6}}

Построим ДИ для параметров

\begin{Shaded}
\begin{Highlighting}[]
\NormalTok{C<-}\KeywordTok{diag}\NormalTok{(}\KeywordTok{c}\NormalTok{(}\DecValTok{1}\NormalTok{,}\DecValTok{1}\NormalTok{,}\DecValTok{1}\NormalTok{))}
\NormalTok{ph<-bht }\CommentTok{#t(C)%*%bhat}
\NormalTok{V<-}\KeywordTok{diag}\NormalTok{(S1) }\CommentTok{# diag(C%*%S1%*%t(C))}
\NormalTok{xa<-}\KeywordTok{qt}\NormalTok{(}\DecValTok{1}\OperatorTok{-}\NormalTok{al}\OperatorTok{/}\DecValTok{2}\NormalTok{,n}\DecValTok{-2}\NormalTok{)}
\NormalTok{s1<-}\KeywordTok{sqrt}\NormalTok{(s2)}
\NormalTok{d<-xa}\OperatorTok{*}\NormalTok{s1}\OperatorTok{*}\KeywordTok{sqrt}\NormalTok{(V)}
\NormalTok{CI<-}\KeywordTok{data.frame}\NormalTok{(}\DataTypeTok{lw=}\NormalTok{ph}\OperatorTok{-}\NormalTok{d,}\DataTypeTok{up=}\NormalTok{ph}\OperatorTok{+}\NormalTok{d)}
\end{Highlighting}
\end{Shaded}

Полученые интервалы:

\begin{verbatim}
##           lw        up
## 1  3.4410260 6.0647164
## 2  0.1622953 3.5408451
## 3 -0.8533866 0.0979088
\end{verbatim}

Доверительный эллипсоид имеет форму

\(A\alpha=\{x,y:((xyz)^T-\beta)^T*(XX^T)^{-1}*((xyz)^T-\beta)\le 1\}\)\\
Где:\\
\(\beta=\)

\begin{verbatim}
##            [,1]
## [1,]  4.7528712
## [2,]  1.8515702
## [3,] -0.3777389
\end{verbatim}

\((XX^T)=\)

\begin{verbatim}
##             [,1]       [,2]        [,3]
## [1,]  0.25594437 -0.3001346  0.07447286
## [2,] -0.30013459  0.4244056 -0.11552266
## [3,]  0.07447286 -0.1155227  0.03364738
\end{verbatim}

\hypertarget{section-7}{%
\subsection{8}\label{section-7}}

Гипотеза линейной регрессионной зависимости Y от X: \(H0: \beta3=0\)\\
Критерий:\\
\(\Phi(x)=\begin{cases} 0, \alpha<PV \\ 1,\alpha>PV \end{cases}\)

Найдем статистику F-критерия и P-значение:

\begin{Shaded}
\begin{Highlighting}[]
\NormalTok{FST<-bht[}\DecValTok{3}\NormalTok{]}\OperatorTok{^}\DecValTok{2}\OperatorTok{/}\NormalTok{S1[}\DecValTok{2}\NormalTok{,}\DecValTok{2}\NormalTok{]}\OperatorTok{/}\NormalTok{s2}
\NormalTok{pv.f<-}\KeywordTok{pf}\NormalTok{(FST,}\DecValTok{1}\NormalTok{,n}\DecValTok{-2}\NormalTok{,}\DataTypeTok{lower.tail=}\OtherTok{FALSE}\NormalTok{)}
\end{Highlighting}
\end{Shaded}

Получим 0.0844295 и 0.7726338\\
Результат: TRUE

\end{document}
