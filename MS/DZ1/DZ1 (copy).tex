% This LaTeX was auto-generated from MATLAB code.
% To make changes, update the MATLAB code and export to LaTeX again.

\documentclass{article}

\usepackage[utf8]{inputenc}
\usepackage[T1]{fontenc}
\usepackage{lmodern}
\usepackage{graphicx}
\usepackage{color}
\usepackage{listings}
\usepackage{hyperref}
\usepackage{amsmath}
\usepackage{amsfonts}
\usepackage{epstopdf}
\usepackage[table]{xcolor}
\usepackage{matlab}

\sloppy
\epstopdfsetup{outdir=./}
\graphicspath{ {./DZ1_images/} }

\begin{document}

\begin{matlabcode}
syms C
syms x
syms y
syms xi
syms eta
syms Sigma
\end{matlabcode}


\begin{matlabcode}

\end{matlabcode}

\begin{par}
\begin{flushleft}
Ларин Антон
\end{flushleft}
\end{par}

\begin{par}
\begin{flushleft}
Вар. 12 (83832020)
\end{flushleft}
\end{par}

\begin{par}
\begin{flushleft}
Плотность двумерного нормального распределения имеет вид:
\end{flushleft}
\end{par}

\begin{matlabcode}

P_xi_eta = C * exp((-1/2)*(2*x^2 + 4*x*y + 5*y^2 - 12*y + 12))
\end{matlabcode}


\begin{enumerate}
\setlength{\itemsep}{-1ex}
   \item{\begin{flushleft} Вычислить вектор мат. ожиданий и ковариационные характеристики данного случайного вектора. \end{flushleft}}
\end{enumerate}

\begin{matlabcode}
(2*x^2 + 4*x*y + 5*y^2 - 12*y + 12)
2*(x+2)^2 + 4*(x+2)*(y-2) + 5*(y-2)^2
\end{matlabcode}


\begin{matlabcode}
E = sym([-2;2])
Sigma_ = [2,2;2,5]
Sigma = Sigma_^(-1)
d = sym(det(Sigma))
cov_xi_xi = sym(Sigma(1,1))
cov_eta_eta = sym(Sigma(2,2))
cov_xi_eta = sym(Sigma(1,2))
Rho = cov_xi_eta * 1/sqrt((cov_xi_xi*cov_eta_eta))
C = 1/(2*pi * sym(sqrt(det(Sigma))))

\end{matlabcode}


\begin{matlabcode}
syms x
syms y
\end{matlabcode}

\begin{par}
\begin{flushleft}
2. Найти аффинное преобразование, переводящее исходный случайный вектор в стандартный нормальный
\end{flushleft}
\end{par}

\begin{matlabcode}
c = [sym(sqrt(2))*x + sym(sqrt(2))*y;sym(sqrt(3))*y - sym(sqrt(3))*2]
vec = [x;y]-E
B = sym([sqrt(2),sqrt(2);0,sqrt(3)])
simplify(c) == simplify(B*vec)
B*Sigma*B.'

\end{matlabcode}


\begin{matlabcode}

\end{matlabcode}

\begin{par}
\begin{flushleft}
3. Найти ортогональное преобразование, переводящее соответствующий центрированный случайный векторв вектор с незывисимыми компонентами
\end{flushleft}
\end{par}

\begin{matlabcode}
Sigma_
[Q,lambda]=eig(Sigma_,"vector");
labmda = sym(lambda)
Q = sym(Q)

D = Q*Sigma*Q.'
sym(det(Sigma))
\end{matlabcode}

\begin{par}
$$\left\lbrack \begin{array}{c}
\xi_1 \\
\xi_2 
\end{array}\right\rbrack ~$$
\end{par}

\begin{matlabcode}
disp('N(')
disp(E/sym(sqrt(5)))
disp(',')
disp(D)
disp(')')

\end{matlabcode}


\begin{par}
\begin{flushleft}
4. Вычислить характеристики совместного распределения случайного вектора ($-2\xi -5\eta ,-5\xi -5\eta$) и записать его плотность
\end{flushleft}
\end{par}

\begin{matlabcode}
B = sym([-2,-5;-5,-5])
E_1 = B*E
Sigma_1 = B*(Sigma)*B.'
Sigma_1_ = Sigma_1^(-1)
x_vec = [sym('x'),sym('y')].'
P = (1/(sqrt(sym(2*pi))*sqrt(det(Sigma_1))))*exp(sym(-1/2)*(x_vec-E_1).'*Sigma_1_*(x_vec-E_1))

\end{matlabcode}


\begin{par}
\begin{flushleft}
5. Найти условное распределение $\xi$ при условии $\eta$
\end{flushleft}
\end{par}

\begin{matlabcode}
P_xi_eta = C * exp((-1/2)*(2*x^2 + 4*x*y + 5*y^2 - 12*y + 12))
P_eta = int(P,sym('x'),-inf,inf)
p_xi_g_eta = P/P_eta 
simplify(p_xi_g_eta)

\end{matlabcode}

\end{document}
