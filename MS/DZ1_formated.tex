% This LaTeX was auto-generated from MATLAB code.
% To make changes, update the MATLAB code and export to LaTeX again.

\documentclass{article}


\usepackage[russian,english]{textcomp}
\usepackage[utf8]{inputenc}
\usepackage[T1,T2A]{fontenc}
\usepackage[UKenglish,russian]{babel}

\usepackage[utf8]{inputenc}
\usepackage[T1]{fontenc}
\usepackage{lmodern}
\usepackage{graphicx}
\usepackage{color}
\usepackage{listings}
\usepackage{hyperref}
\usepackage{amsmath}
\usepackage{amsfonts}
\usepackage{epstopdf}
\usepackage[table]{xcolor}
\usepackage{matlab}

\sloppy
\epstopdfsetup{outdir=./}
\graphicspath{ {./DZ1_images/} }

\begin{document}

\begin{par}
\begin{flushleft}
Ларин Антон
\end{flushleft}
\end{par}

\begin{par}
\begin{flushleft}
Вар. 12 (83832020)
\end{flushleft}
\end{par}

%\begin{matlabcode}
%syms C
%syms x
%syms y
%syms xi
%syms eta
%syms Sigma
%\end{matlabcode}



\begin{par}
\begin{flushleft}
Плотность двумерного нормального распределения имеет вид:
\end{flushleft}
\end{par}

%\begin{matlabcode}
%P_xi_eta = C * exp((-1/2)*(2*x^2 + 4*x*y + 5*y^2 - 12*y + 12))
%\end{matlabcode}
\begin{matlabsymbolicoutput}
$P_{\xi\ \eta} = \displaystyle C\,{\mathrm{e}}^{-x^2 -2\,x\,y-\frac{5\,y^2 }{2}+6\,y-6} $
\end{matlabsymbolicoutput}


\begin{enumerate}
\setlength{\itemsep}{-1ex}
   \item{\begin{flushleft} Вычислить вектор мат. ожиданий и ковариационные характеристики данного случайного вектора. \end{flushleft}}
   %\item{\begin{flushleft} q. \end{flushleft}}
\end{enumerate}

%\begin{matlabcode}
%(2*x^2 + 4*x*y + 5*y^2 - 12*y + 12)
%\end{matlabcode}
\begin{matlabsymbolicoutput}
    $\displaystyle 2\,x^2 +4\,x\,y+5\,y^2 -12\,y+12$
\end{matlabsymbolicoutput}
%\begin{matlabcode}
%2*(x+2)^2 + 4*(x+2)*(y-2) + 5*(y-2)^2
%\end{matlabcode}
\begin{matlabsymbolicoutput}
    $\displaystyle 2\,{{\left(x+2\right)}}^2 +5\,{{\left(y-2\right)}}^2 +{\left(4\,x+8\right)}\,{\left(y-2\right)}$
\end{matlabsymbolicoutput}


%\begin{matlabcode}
%E = sym([-2;2])
%\end{matlabcode}
\begin{matlabsymbolicoutput}
E = 
    $\displaystyle \left(\begin{array}{c}
-2\\
2
\end{array}\right)$
\end{matlabsymbolicoutput}
%\begin{matlabcode}
%Sigma_ = [2,2;2,5]
%\end{matlabcode}

\begin{matlabsymbolicoutput}
$\Sigma^{-1} =$
     2     2
     2     5

\end{matlabsymbolicoutput}
%\begin{matlabcode}
%Sigma = Sigma_^(-1)
%\end{matlabcode}
\begin{matlabsymbolicoutput}
$\Sigma =$ 
    0.8333   -0.3333
   -0.3333    0.3333

\end{matlabsymbolicoutput}
%\begin{matlabcode}
%d = sym(det(Sigma))
%\end{matlabcode}
\begin{matlabsymbolicoutput}
$det(\Sigma) = $
    $\displaystyle \frac{1}{6}$
\end{matlabsymbolicoutput}
%in{matlabcode}
%cov_xi_xi = sym(Sigma(1,1))
%\end{matlabcode}
\begin{matlabsymbolicoutput}
$cov_{\xi\ \xi} = $
    $\displaystyle \frac{5}{6}$
\end{matlabsymbolicoutput}
%\begin{matlabcode}
%cov_eta_eta = sym(Sigma(2,2))
%\end{matlabcode}
\begin{matlabsymbolicoutput}
$cov_{\eta\ \eta} = $ 
    $\displaystyle \frac{1}{3}$
\end{matlabsymbolicoutput}
%\begin{matlabcode}
%cov_xi_eta = sym(Sigma(1,2))
%\end{matlabcode}
\begin{matlabsymbolicoutput}
$cov_{\xi\ \eta} = $ 
    $\displaystyle -\frac{1}{3}$
\end{matlabsymbolicoutput}
%\begin{matlabcode}
%Rho = cov_xi_eta * 1/sqrt((cov_xi_xi*cov_eta_eta))
%\end{matlabcode}
\begin{matlabsymbolicoutput}
$\rho = cov_{\xi\ \eta} * \frac{1}{\sqrt{cov_{\xi\ \xi}*cov_{\eta\ \eta}))}} =$
    $\displaystyle -\frac{\sqrt{2}\,\sqrt{5}}{5}$
\end{matlabsymbolicoutput}
%\begin{matlabcode}
%C = 1/(2*pi * sym(sqrt(det(Sigma))))
%\end{matlabcode}
\begin{matlabsymbolicoutput}
$C = \frac{1}{2*\pi * \sqrt{det(\Sigma)}} =$
    $\displaystyle \frac{\sqrt{6}}{2\,\pi }$
\end{matlabsymbolicoutput}
%
%\begin{matlabcode}
%syms x
%syms y
%\end{matlabcode}

\begin{par}
\begin{flushleft}
2. Найти аффинное преобразование, переводящее исходный случайный вектор в стандартный нормальный
\end{flushleft}
\end{par}

%\begin{matlabcode}
%c = [sym(sqrt(2))*x + sym(sqrt(2))*y;sym(sqrt(3))*y - sym(sqrt(3))*2]
%\end{matlabcode}
\begin{matlabsymbolicoutput}
c = 
    $\displaystyle \left(\begin{array}{c}
\sqrt{2}\,x+\sqrt{2}\,y\\
\sqrt{3}\,y-2\,\sqrt{3}
\end{array}\right)$
\end{matlabsymbolicoutput}
%\begin{matlabcode}
%vec = [x;y]-E
%\end{matlabcode}
\begin{matlabsymbolicoutput}
vec = 
    $\displaystyle \left(\begin{array}{c}
x+2\\
y-2
\end{array}\right)$
\end{matlabsymbolicoutput}
\begin{matlabsymbolicoutput}
B = 
    $\displaystyle \left(\begin{array}{cc}
\sqrt{2} & \sqrt{2}\\
0 & \sqrt{3}
\end{array}\right)$
\end{matlabsymbolicoutput}
\begin{matlabsymbolicoutput}
B * vec = c
\end{matlabsymbolicoutput}
%\begin{matlabcode}
%B = sym([sqrt(2),sqrt(2);0,sqrt(3)])
%\end{matlabcode}

%\begin{matlabcode}
%simplify(c) == simplify(B*vec)
%\end{matlabcode}
\begin{matlabsymbolicoutput}
    $\displaystyle \left(\begin{array}{c}
\sqrt{2}\,x+\sqrt{2}\,y=\sqrt{2}\,{\left(x+y\right)}\\
\sqrt{3}\,y-2\,\sqrt{3}=\sqrt{3}\,{\left(y-2\right)}
\end{array}\right)$
\end{matlabsymbolicoutput}
%\begin{matlabcode}
%B*Sigma*B.'
%\end{matlabcode}
\begin{matlabsymbolicoutput}
$B*\Sigma*B^{T} =$
    $\displaystyle \left(\begin{array}{cc}
1 & 0\\
0 & 1
\end{array}\right)$
\end{matlabsymbolicoutput}

\begin{par}
\begin{flushleft}
3. Найти ортогональное преобразование, переводящее соответствующий центрированный случайный векторв вектор с незывисимыми компонентами
\end{flushleft}
\end{par}

%\begin{matlabcode}
%Sigma_
%\end{matlabcode}
\begin{matlabsymbolicoutput}
$\Sigma^{-1} =$    
     2     2
     2     5
\end{matlabsymbolicoutput}
\begin{matlabcode}
[Q,lambda]=eig(Sigma_,"vector");
\end{matlabcode}
\begin{matlabsymbolicoutput}
$\lambda =$
    $\displaystyle \left(\begin{array}{c}
1\\
6
\end{array}\right)$
\end{matlabsymbolicoutput}
%\begin{matlabcode}
%Q = sym(Q)
%\end{matlabcode}
\begin{matlabsymbolicoutput}
Q = 
    $\displaystyle \left(\begin{array}{cc}
-\frac{2\,\sqrt{5}}{5} & \frac{\sqrt{5}}{5}\\
\frac{\sqrt{5}}{5} & \frac{2\,\sqrt{5}}{5}
\end{array}\right)$
\end{matlabsymbolicoutput}
%\begin{matlabcode}
%D = Q*Sigma*Q.'
%\end{matlabcode}
\begin{matlabsymbolicoutput}
$D = Q*\Sigma*Q^{T} =$
    $\displaystyle \left(\begin{array}{cc}
1 & 0\\
0 & \frac{1}{6}
\end{array}\right)$
\end{matlabsymbolicoutput}
%\begin{matlabcode}
%sym(det(Sigma))
%\end{matlabcode}
%\begin{matlabsymbolicoutput}
%ans = 
%    $\displaystyle \frac{1}{6}$
%\end{matlabsymbolicoutput}

\begin{par}
    $\displaystyle \left(\begin{array}{c}
\xi_1 \\
\xi_2 
\end{array}\right) \sim N \displaystyle 
\left(
\displaystyle \left(\begin{array}{c}
-\frac{2\,\sqrt{5}}{5}\\
\frac{2\,\sqrt{5}}{5}
\end{array}\right),
\displaystyle \left(\begin{array}{cc}
1 & 0\\
0 & \frac{1}{6}
\end{array}\right)
\right)
$
\end{par}

%\begin{matlabcode}
%disp('N(')
%\end{matlabcode}
%\begin{matlaboutput}
%N(
%\end{matlaboutput}
%\begin{matlabcode}
%disp(E/sym(sqrt(5)))
%\end{matlabcode}
%\begin{matlabsymbolicoutput}
%$\displaystyle \left(\begin{array}{c}
%-\frac{2\,\sqrt{5}}{5}\\
%\frac{2\,\sqrt{5}}{5}
%\end{array}\right)$
%\end{matlabsymbolicoutput}
%\begin{matlabcode}
%disp(',')
%\end{matlabcode}
%\begin{matlaboutput}
%,
%\end{matlaboutput}
%\begin{matlabcode}
%disp(D)
%\end{matlabcode}
%\begin{matlabsymbolicoutput}
%$\displaystyle \left(\begin{array}{cc}
%1 & 0\\
%0 & \frac{1}{6}
%\end{array}\right)$
%\end{matlabsymbolicoutput}
%\begin{matlabcode}
%disp(')')
%\end{matlabcode}
%\begin{matlaboutput}
%)
%\end{matlaboutput}


\begin{par}
\begin{flushleft}
4. Вычислить характеристики совместного распределения случайного вектора ($-2\xi -5\eta ,-5\xi -5\eta$) и записать его плотность
\end{flushleft}
\end{par}

%\begin{matlabcode}
%B = sym([-2,-5;-5,-5])
%\end{matlabcode}
\begin{matlabsymbolicoutput}
B = 
    $\displaystyle \left(\begin{array}{cc}
-2 & -5\\
-5 & -5
\end{array}\right)$
\end{matlabsymbolicoutput}
%\begin{matlabcode}
%E' = B*E
%\end{matlabcode}
\begin{matlabsymbolicoutput}

E' = B*E =
    $\displaystyle \left(\begin{array}{c}
-6\\
0
\end{array}\right)$
\end{matlabsymbolicoutput}
%\begin{matlabcode}
%Sigma_1 = B*(Sigma)*B.'
%\end{matlabcode}
\begin{matlabsymbolicoutput}
$\Sigma' =  B*\Sigma*B^{T} =$
    $\displaystyle \left(\begin{array}{cc}
5 & 5\\
5 & \frac{25}{2}
\end{array}\right)$
\end{matlabsymbolicoutput}
\begin{matlabcode}
Sigma_1_ = Sigma_1^(-1)
\end{matlabcode}
\begin{matlabsymbolicoutput}
$\Sigma'^{-1} =$
    $\displaystyle \left(\begin{array}{cc}
\frac{1}{3} & -\frac{2}{15}\\
-\frac{2}{15} & \frac{2}{15}
\end{array}\right)$
\end{matlabsymbolicoutput}
%\begin{matlabcode}
%$\overline{x} = [sym('x'),sym('y')].'$
%\end{matlabcode}
\begin{matlabsymbolicoutput}
$\overline{x} = $
    $\displaystyle \left(\begin{array}{c}
x\\
y
\end{array}\right)$
\end{matlabsymbolicoutput}
%\begin{matlabcode}
%P = (1/(sqrt(sym(2*pi))*sqrt(det(Sigma_1))))*exp(sym(-1/2)*(x_vec-E_1).'*Sigma_1_*(x_vec-E_1))
%\end{matlabcode}
\begin{matlabsymbolicoutput}
$P = \left(\frac{1}{(\sqrt{2*pi}*\sqrt{det(\Sigma')}}\right)*e^{-\frac{1}{2}*(\overline{x}-E')^{-1}*\Sigma'^{-1}*(\overline{x}-E')} =$
    $\displaystyle \frac{\sqrt{3}\,{\mathrm{e}}^{y\,{\left(\frac{x}{15}-\frac{y}{15}+\frac{2}{5}\right)}-{\left(x+6\right)}\,{\left(\frac{x}{6}-\frac{y}{15}+1\right)}} }{15\,\sqrt{\pi }}$
\end{matlabsymbolicoutput}


\begin{par}
\begin{flushleft}
5. Найти условное распределение $\xi$ при условии $\eta$
\end{flushleft}
\end{par}

%\begin{matlabsymbolicoutput}
%$P_xi_eta = C * exp((-1/2)*(2*x^2 + 4*x*y + 5*y^2 - 12*y + 12))$
%\end{matlabsymbolicoutput}
\begin{matlabsymbolicoutput}
$P_{\xi\ \eta} = $
    $\displaystyle \frac{\sqrt{6}}{2\,\pi }*{\mathrm{e}}^{-x^2 -2\,x\,y-\frac{5\,y^2 }{2}+6\,y-6} $
\end{matlabsymbolicoutput}
\begin{matlabcode}
P_eta = int(P,sym('x'),-inf,inf)
\end{matlabcode}
\begin{matlabsymbolicoutput}
$P_{\eta} = \Large \displaystyle \int\limits_{-\infty}^{\infty} P_{\xi\ \eta} dx =$
    $\displaystyle \frac{\sqrt{2}}{5} * {\mathrm{e}}^{-\frac{y^2 }{25}}$
\end{matlabsymbolicoutput}
%\begin{matlabcode}
%p_xi_g_eta = P/P_eta 
%\end{matlabcode}
\begin{matlabsymbolicoutput}
$P_{\xi|\eta} = \frac{P_{\xi\ \eta}}{P_{\eta}} =$
    $\displaystyle \frac{1}{\sqrt{6\pi}} * {\mathrm{e}}^{\frac{y^2 }{25} + y\,{\left(\frac{x}{15}-\frac{y}{15}+\frac{2}{5}\right)}-{\left(x+6\right)}\,{\left(\frac{x}{6}-\frac{y}{15}+1\right)}} = $
\end{matlabsymbolicoutput}
%\begin{matlabcode}
%simplify(p_xi_g_eta)
%\end{matlabcode}
\begin{matlabsymbolicoutput}
    $\displaystyle \frac{1 }{\sqrt{6\pi}}*{\mathrm{e}}^{-\frac{{{\left(5\,x-2\,y+30\right)}}^2 }{150}}$
\end{matlabsymbolicoutput}
%\begin{matlabcode}
%
%\end{matlabcode}

\end{document}

